%Klimbim für die Formatierung
\documentclass[12pt]{article}
\usepackage[a4paper,naturalnames]{hyperref}
\usepackage[utf8]{inputenc}  
\usepackage[german]{babel}
\usepackage{setspace}
\usepackage{fancyhdr}
\usepackage{url}
\usepackage{lastpage} % Für Seitenanzahl
\usepackage{calc}

%Tabellen
\usepackage{tabularx}
\usepackage{paralist} % Für die Aufzählungen in den Tabellen, ermöglicht kurzen abstand
\usepackage{longtable}
%%%%%%%%%%%%%%%%%%%%%%%%%%%%%%
\setlength{\parskip}{0.5ex}%
\setlength{\parindent}{0pt}%
%\renewcommand{\baselinestretch}{1.5}%
%

\oddsidemargin -10.4mm
\evensidemargin -10.4mm
\textwidth 18cm
\headheight 20mm
\topmargin -23mm
\textheight 222mm
\setlength{\tabcolsep}{0.25cm}
\setlength{\arrayrulewidth}{1pt}

%%%%%%%%%%%%%%%%%%%%%%%%%%%%%%

%Kopf und Fusszeilen
\pagestyle{fancy}
\newcommand{\tstamp}{DATUM}
\fancyhf{}
\lhead{\sf J+S-NUMMER}
\chead{BLOCKNAME}
\rhead{\sf LOGOS J+S, JUBLA ZUFIKON}
\cfoot{\thepage / \pageref{LastPage}}
\rfoot{\sf \tstamp}
\lfoot{\sf Sommerlager 2010 Stoos}
\renewcommand{\headrulewidth}{1pt}
\renewcommand{\footrulewidth}{1pt}

%Zeilenabstand 1.5
\onehalfspacing 
\begin{document}
\section*{Blockbeschrieb}
	\begin{tabular}{|p{2.5cm}|p{5.5cm}|p{2.5cm}|p{5.5cm}|}
	\hline
	Blockname: 	& BLOCKNAME	& J+S-Nummer	& J+S-NUMMER \\ \hline
	Leitung:		& LEITUNG		& Teilnehmer:	& TEILNEHMER \\ \hline
	Ort:			& ORT			& Datum:		& \tstamp \\ \hline
	Material:		& MATERIAL		& Zeit:		& ZEIT \\ \hline
	Ziele:		&  \multicolumn{3}{p{13.5cm+4\tabcolsep+2\arrayrulewidth}|}{\begin{compactitem} % Compactitem ergibt eine Liste
		\item ZIELE % Hier sind die blöden Umbrüche
 	\end{compactitem}} \\
	\hline
\end{tabular}



\begin{tabular}{|p{17.5cm+3\arrayrulewidth}|}
	\hline
	\parbox[0pt][1.5cm][c]{0cm}{\section*{Sicherheitskonzept}}\\
	\hline
	\begin{compactitem}
		 \item SICHERHEITSKONZEPT
 	\end{compactitem}\\
 	\hline 
\end{tabular}

\begin{longtable}{|p{12cm}|p{5cm+2\arrayrulewidth}|}
	\hline 
	\bf Blockbeschrieb & \bf Bemerkungen \\ \hline 
	\endfirsthead \hline \bf Blockbeschrieb & \bf Bemerkungen \\ \hline 
	\endhead \hline 
	\multicolumn{2}{|c|}{Fortsetzung auf nächster Seite.} \\ \hline 
	\endfoot
	\endlastfoot 
	{\large\bf {Einstieg}}
	
	EINSTIEG & \\ \hline
	{\large\bf {Hauptteil}}
	
	HAUPTTEIL & \\ \hline
	{\large\bf {Ausstieg}}
	
	AUSSTIEG & \\ \hline

	\end{longtable}
\end{document}